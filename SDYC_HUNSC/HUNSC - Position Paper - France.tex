\documentclass[a4paper,12pt]{article}
\usepackage[utf8]{inputenc}
% \usepackage{tgtermes}
\usepackage{graphicx}
\usepackage{geometry}
\usepackage[style=apa, backend=biber]{biblatex}
\usepackage{setspace}

\geometry{margin=1in}
\setlength{\parindent}{2em}
\setlength{\parskip}{1em}
\setstretch{1}
\interfootnotelinepenalty=10000

\DeclareLanguageMapping{english}{english-apa}

\bibliography{refs.bib}

\title{
	\vspace{0.7in}
	United Nations Security Council (Historical)\\
	\vspace{1em}
	\textbf{The French Fifth Republic} \\
	\vspace{0.5em}
	\textbf{The Korean War}
	\vspace{4in}
}

\date{
	\vspace{0in}
	June 25, 1950
}

\author{
	Erica Neu\\ Peng Peiyao \\
}

\begin{document}

\maketitle

\clearpage


As the Korean Peninsula faces a critical juncture with the outbreak of the Korean War on June 25, 1950 (\cite{imperialwarmuseums_2024_a}), France is deeply concerned about the aggressive actions taken by North Korea against the Republic of Korea. The invasion of South Korea by North Korean forces under the command of Kim Il-sung represents a clear violation of international law and the principles of sovereignty and territorial integrity enshrined in the United Nations Charter (\cite{nations_2023_united}). The Korean Peninsula's strategic significance, particularly in the context of the broader Cold War tensions, is undeniable. France, as a member of the United Nations Security Council, is committed to upholding international peace and security. Given our nation's experience in rebuilding from the devastation of World War II, France understands the catastrophic consequences of unchecked military aggression and the importance of a collective international response.

France, pursuant to its obligations under the United Nations Charter, unequivocally condemns North Korea's invasion of South Korea and calls for the immediate cessation of any and all hostilities, and stands in solidarity with the Republic of Korea and the Korean people in their struggle to defend their sovereignty and independence. The French Government believes that such aggression not more than half a decade after the end of World War II (\cite{britannica_2024_world}) is nothing short of highly appalling, and is wholeheartedly committed to supporting the United Nations Security Council in taking decisive action to address this crisis and restore peace and stability to the Korean Peninsula.

France would therefore like to call for the immediate and complete withdrawal of all North Korean forces from any territory beyond the established 38th parallel (\cite{thenationalendowmentforthehumanities_2019_korea}), and the establishment of a United Nations Command (UNC) to coordinate and support military efforts in the defense of South Korea and if necessary, the liberation of North Korea from the oppressive regime of Kim Il-sung. France also calls for the imposition of economic and diplomatic sanctions and other measures to isolate North Korea and hold its leadership accountable for their actions.

Should North Korean refuse to cease and desist their participation in such nefarious acts, France calls for the aforementioned UNC to immediately create and deploy a Quick Response Force\footnotemark[1] to the Korean Peninsula to halt the North Korean advance and ensure the safety of South Korean people. France, as a member of the North Atlantic Treaty Organization (NATO), is prepared to contribute military personnel and equipment to the UNC and QRF, and calls on other member states of the United Nations Security Council to do the same. France also advocates for the establishment of a United Nations Commission of Inquiry to investigate and document the human rights abuses and war crimes committed by North Korean forces, and to hold those responsible accountable for their actions.

France calls upon the United States to take the lead in coordinating and commanding the UNC and QRF, and to work closely with France and other member states of the United Nations Security Council to develop a comprehensive strategy to address the crisis on the Korean Peninsula. France would also like to call upon the Soviet Union and China to use their influence with North Korea to persuade the regime to comply with the demands of the United Nations Security Council and to avoid further escalation of the conflict.

\newpage

\footnotetext[1]{
	Abbreviated as QRF.\@ Should optimally be a unit of battalion to brigade strength, composed of a multinational force placed under a single unified command structure to facilitate rapid deployment and response to events. It is recommended to be a self sustaining force capable of operating independently for a limited period of time, comprising of infantry, armor, artillery, and air support units, allowing for a flexible and adaptable response to a wide range of contingencies without the need to requisition additional resources and support from other units or commands when deployed on short notice. Further support or reinforcements can be requested from other units or commands as necessary, but the QRF should be capable of operating independently until such support arrives.
	\\
	France calls for the US to take the lead in commanding the QRF and providing the necessary equipment, armaments or otherwise, to simplify logistical support and ensure the rapid deployment and effectiveness of the force. France and other member states of the United Nations Security Council should provide additional personnel and equipment as necessary to augment the QRF and ensure its success in achieving its objectives. The QRF should be prepared to deploy to the Korean Peninsula at short notice and be capable of operating in a wide range of environments and conditions to respond to any contingency that may arise during the conflict. It is recommended that the QRF be equipped with the latest military technology and equipment to ensure its maximum effectiveness and prolonged capability to operate independently in hostile environments.
	\\
	The QRF's main ground component should be composed of infantry units as its main fighting component. It is to be supplemented with calvary and armoured units, optimally equipped with M26 Pershing Medium Tanks, due to the ineffectiveness of the 75mm gun M3 and M6 fitted on the M4 Shermans and M24 Chaffees respectively against the T-34/85s and IS-2s fielded by the North Koreans (\cite{bird2001world}). The 90mm gun M3 fitted on the M26 Pershing is capable of penetrating the frontal armour of the T-34/85 and IS-2 at combat ranges (\cite{bird2001world}), providing the QRF with a significant advantage in firepower and protection against North Korean armour. The QRF is also to be equipped with self propelled artillery units to provide fire support and mobility to the force preferably with the likes of the M7 Priest, M37 and M41 Howitzer Motor Carriage (HMC), pending availability and confirmation. Usage of towed artillery is discouraged due to the need for rapid deployment and redeployment of the force, and the potential for logistical difficulties in transporting and resupplying towed artillery units in a combat environment. Artillery fire support may also be requisitioned from US Navy (USN) warships stationed off the coast of Korea or other nearby land-based artillery units to supplement the QRF's own artillery units and provide additional firepower and support to the force if necessary.
	\\
	The QRF's air wing should be capable of providing air superiority and air interdiction missions to support the QRF's ground operations and to deny North Korean forces the ability to operate freely in the air. It is recommended that the QRF be supported by a dedicated air wing in the form of an attaché embedded in the command structure of land or naval-based air units situated in relative close proximity to the QRF's area of operations, where the attaché exercises sole or major jurisdiction over a dedicated section of the air unit and answers directly to QRF command. Close air support missions can be carried out by F-51D Mustangs and the F-82 Twin Mustangs from the US Air Force (USAF), F4U/AU-1 Corsairs, A-1 Skyraiders and F9F Panthers from the US Marine Corps (USMC) and USN. F-80C Shooting Stars from the USAF and F9F Panthers and F2H Banshees from the USN can be used for combat air patrol and maintaining air superiority over the QRF's area of operations. C-47 Skytrains and C-54 Skymasters can be used for transport and resupply missions to support the QRF's ground operations, through airdrops or otherwise. C-47s can also be utilised as `flare ships' to provide illumination for night operations by dropping magnesium flares over the area of operations. The air wing should also be capable of conducting reconnaissance and surveillance missions to provide the QRF with real-time intelligence on enemy movements and positions, and to facilitate the coordination of air and ground operations to achieve the QRF's objectives.
}

\clearpage

\printbibliography

\end{document}
