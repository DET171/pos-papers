\documentclass[a4paper,12pt]{article}
\usepackage[utf8]{inputenc}
% \usepackage{tgtermes}
\usepackage{graphicx}
\usepackage{geometry}
\usepackage[style=apa, backend=biber]{biblatex}
\usepackage{setspace}

\geometry{margin=1in}
\setlength{\parindent}{2em}
\setlength{\parskip}{1em}
\setstretch{1}
\interfootnotelinepenalty=10000

\DeclareLanguageMapping{english}{english-apa}

\bibliography{refs.bib}

\title{
	\vspace{0.7in}
	United Nations Security Council (Historical)\\
	\vspace{1em}
	\textbf{The French Fifth Republic} \\
	\vspace{0.5em}
	\textbf{The Korean War}
	\vspace{4in}
}

\date{
	\vspace{0in}
	June 25, 1950
}

\author{
	Erica Neu\\ Peng Peiyao \\
}

\begin{document}

\maketitle

\clearpage


As the Korean Peninsula faces a critical juncture with the outbreak of the Korean War on June 25, 1950 (\cite{imperialwarmuseums_2024_a}), France is deeply concerned about the aggressive actions taken by North Korea against the Republic of Korea. The invasion of South Korea by North Korean forces under the command of Kim Il-sung represents a clear violation of international law and the principles of sovereignty and territorial integrity enshrined in the United Nations Charter (\cite{nations_2023_united}). The Korean Peninsula's strategic significance, particularly in the context of the broader Cold War tensions, is undeniable. France, as a member of the United Nations Security Council, is committed to upholding international peace and security. Given our nation's experience in rebuilding from the devastation of World War II, France understands the catastrophic consequences of unchecked military aggression and the importance of a collective international response.

France, pursuant to its obligations under the United Nations Charter, unequivocally condemns North Korea's invasion of South Korea and calls for the immediate cessation of any and all hostilities, and stands in solidarity with the Republic of Korea and the Korean people in their struggle to defend their sovereignty and independence. The French Government believes that such aggression not more than half a decade after the end of World War II (\cite{britannica_2024_world}) is nothing short of highly appalling, and is wholeheartedly committed to supporting the United Nations Security Council in taking decisive action to address this crisis and restore peace and stability to the Korean Peninsula.

France would therefore like to call for the immediate and complete withdrawal of all North Korean forces from any territory beyond the established 38th parallel (\cite{thenationalendowmentforthehumanities_2019_korea}), and the establishment of a United Nations Command (UNC) to coordinate and support military efforts in the defense of South Korea and if necessary, the liberation of North Korea from the oppressive regime of Kim Il-sung. France also calls for the imposition of economic and diplomatic sanctions and other measures to isolate North Korea and hold its leadership accountable for their actions.

Should North Korean refuse to cease and desist their participation in such nefarious acts, France calls for the aforementioned UNC to immediately create and deploy a Quick Response Force\footnotemark[1]\textsuperscript{,}\footnotemark[2] (refer to footnote 1 for more details, footnote 2 for proposed composition) to the Korean Peninsula to halt the North Korean advance and ensure the safety of South Korean people. France, as a member of the North Atlantic Treaty Organization (NATO), is prepared to contribute military personnel and equipment to the UNC and QRF, and calls on other member states of the United Nations Security Council to do the same. France also advocates for the establishment of a United Nations Commission of Inquiry to investigate and document the human rights abuses and war crimes committed by North Korean forces, and to hold those responsible accountable for their actions.

France calls upon the United States to take the lead in coordinating and commanding the UNC and QRF, and to work closely with France and other member states of the United Nations Security Council to develop a comprehensive strategy to address the crisis on the Korean Peninsula. France would also like to call upon the Soviet Union and China to use their influence with North Korea to persuade the regime to comply with the demands of the United Nations Security Council and to avoid further escalation of the conflict.

\newpage

\footnotetext[1]{
	Abbreviated as QRF.\@ Should optimally be a unit of battalion to brigade strength, composed of a multinational force placed under a single unified command structure to facilitate rapid deployment and response to events. It is recommended to be a self sustaining force capable of operating independently for a limited period of time, comprising of infantry, armor, artillery, and air support units, allowing for a flexible and adaptable response to a wide range of contingencies without the need to requisition additional resources and support from other units or commands when deployed on short notice. Further support or reinforcements can be requested from other units or commands as necessary, but the QRF should be capable of operating independently until such support arrives.
	\\
	France calls for the US to take the lead in commanding the QRF and providing the necessary equipment, armaments or otherwise, to simplify logistical support and ensure the rapid deployment and effectiveness of the force. France and other member states of the United Nations Security Council should provide additional personnel and equipment as necessary to augment the QRF and ensure its success in achieving its objectives. The QRF should be prepared to deploy to the Korean Peninsula at short notice and be capable of operating in a wide range of environments and conditions to respond to any contingency that may arise during the conflict. It is recommended that the QRF be equipped with the latest military technology and equipment to ensure its maximum effectiveness and prolonged capability to operate independently in hostile environments.
	\\
	The QRF's main ground component should be composed of infantry units as its main fighting component. It comprises of a reinforced mechanized infantry regiment (3 battalions) and is to be equipped with M3 Half-Tracks for troop transport, M8 Greyhound Armored Cars for reconnaissance and light support, and M29 Weasels for logistical transport in difficult terrain. It is to be supplemented with calvary and armoured units, optimally equipped with M26 Pershing Medium Tanks, due to the ineffectiveness of the 75mm gun M3 and M6 fitted on the M4 Shermans and M24 Chaffees respectively against the T-34/85s and IS-2s fielded by the North Koreans (\cite{bird2001world}). The 90mm gun M3 fitted on the M26 Pershing is capable of penetrating the frontal armour of the T-34/85 and IS-2 at combat ranges (\cite{bird2001world}), providing the QRF with a significant advantage in firepower and protection against North Korean armour. The QRF is also to be equipped with self propelled artillery units to provide fire support and mobility to the force preferably with the likes of the M7 Priest, M37 and M41 Howitzer Motor Carriage (HMC), pending availability and confirmation. Usage of towed artillery is discouraged due to the need for rapid deployment and redeployment of the force, and the potential for logistical difficulties in transporting and resupplying towed artillery units in a combat environment. Towed artillery pieces possess the inherent disadvantage of being less mobile and more vulnerable to enemy fire compared to self-propelled artillery units, which can move under their own power and provide direct fire support to the force without the need for external transport or support. Artillery fire support may also be requisitioned from US Navy (USN) warships stationed off the coast of Korea or other nearby land-based artillery units to supplement the QRF's own artillery units and provide additional firepower and support to the force if necessary.
	\\
	The QRF's air wing should be capable of providing air superiority and air interdiction missions to support the QRF's ground operations and to deny North Korean forces the ability to operate freely in the air. It is recommended that the QRF be supported by a dedicated air wing in the form of an attaché embedded in the command structure of land or naval-based air units situated in relative close proximity to the QRF's area of operations, where the attaché exercises sole or major jurisdiction over a dedicated section of the air unit and answers directly to QRF command. Close air support missions can be carried out by F-51D Mustangs and the F-82 Twin Mustangs from the US Air Force (USAF), F4U/AU-1 Corsairs, A-1 Skyraiders and F9F Panthers from the US Marine Corps (USMC) and USN. F-80C Shooting Stars from the USAF and F9F Panthers and F2H Banshees from the USN can be used for combat air patrol and maintaining air superiority over the QRF's area of operations. C-47 Skytrains and C-54 Skymasters can be used for transport and resupply missions to support the QRF's ground operations, through airdrops or otherwise. C-47s can also be utilised as `flare ships' to provide illumination for night operations by dropping magnesium flares over the area of operations. The air wing should also be capable of conducting reconnaissance and surveillance missions to provide the QRF with real-time intelligence on enemy movements and positions, and to facilitate the coordination of air and ground operations to achieve the QRF's objectives.
}

\clearpage

\footnotetext[2]{
	The QRF is composed of four main elements: the Ground Combat Element (GCE), Air Combat Element (ACE), Logistics Combat Element (LCE), and a Command Element (CE):
	\\
	\textbf{Ground Combat Element (GCE)}: The GCE is the backbone of the QRF's fighting component and is composed of infantry, cavalry, and armoured units. The GCE is responsible for conducting ground operations, engaging enemy forces, and securing objectives:
	\begin{itemize}
		\item Infantry:
		\begin{itemize}
			\item 1x Mechanized Infantry Regiment (3x Mechanized Infantry Battalions), Reinforced (approx. 3,700 - 4,200 personnel)
			\begin{itemize}
				\item 100x M3 Half-Tracks (for troop transport and support)
				\item 15x M8 Greyhound Armored Cars (for reconnaissance and light support)
				\item 10x M4 Sherman Medium Tanks (for armored support, independent of the Armored Battalion)
				\item 15x M29 Weasel (for logistical support in difficult terrain)
			\end{itemize}
			\item 3,200x M1 Garand Rifles
			\item 1,500x M1 Carbines
			\item 300x M1918 Browning Automatic Rifles (BAR)
			\item 150x M1919 Browning Machine Guns
			\item 90x M2 Mortars
			\item 70x M2 Heavy Browning Machine Guns
			\item 60x M9A1 Bazookas
			\item 60x M20 Recoilless Rifles
			\item 30x M2 Flamethrowers
		\end{itemize}
		
		\item Armour:
		\begin{itemize}
			\item 1x Armoured Battalion (approx. 900 - 1,100 personnel)
			\item 80x M26 Pershing Medium Tanks
			\item 40x M4A3E8 Sherman Medium Tanks (supplementary roles)
			\item 20x M24 Chaffee Light Tanks (reconnaissance roles)
		\end{itemize}
		\item Cavalry:
		\begin{itemize}
			\item 1x Cavalry Company (approx. 200 - 250 personnel)
			\item 20x M8/M20 Greyhound Armoured Cars
			\item 15x M39 Armoured Utility Vehicles
		\end{itemize}
		\item Artillery:
		\begin{itemize}
			\item 1x Artillery Battalion (approx. 1,200 personnel)
			\item 42x M7 Priest Self-Propelled Howitzers (SPH)
			\item 24x M37 Howitzer Motor Carriages (HMC)
			\item 12x M41 Howitzer Motor Carriages (HMC)
		\end{itemize}
	\end{itemize}
}

\clearpage

\footnotetext[2]{
	cont'd.
	\\
	\textbf{Air Combat Element (ACE)}: The ACE is responsible for providing air support, air superiority, and air interdiction missions to support the GCE's ground operations and to deny enemy forces the ability to operate freely in the air:
	\begin{itemize}
		\item 1x Air Wing (approx. 1,500 - 1,800 personnel)
		\item 36x F-80C Shooting Stars (air superiority/close air support)
		\item 24x F9F Panthers (air superiority/close air support)
		\item 24x F-51D Mustangs (close air support)
		\item 24x AD-4 Skyraiders (close air support)
		\item 20x H-19 Chickasaw Helicopters (transport and medevac)
		\item 18x F2H Banshees (air superiority/close air support)
		\item 18x F4U-4/AU-1 Corsairs (close air support)
		\item 12x F-82 Twin Mustangs (night fighter/interceptor/close air support)
		\item 18x C-47 Skytrains (resupply, airdrop and flare missions)
		\item 5x C-54 Skymasters
	\end{itemize}
	\textbf{Logistics Combat Element (LCE)}: The LCE is responsible for providing logistical support to the GCE and ACE, including transportation, resupply, medical, and maintenance services:
	\begin{itemize}
		\item 1x Logistics Battalion (approx. 1,200 - 1,400 personnel)
			\begin{itemize}
				\item Motor Transport Company:
				\begin{itemize}
					\item 100x 2.5-ton GMC CCKW Trucks (for general transport)
					\item 20x 5-ton M54 Trucks (for heavy transport and equipment towing)
					\item 15x M29 Weasel Cargo Carriers (for logistical support in difficult terrain)
					\item 10x M4 High-Speed Tractors (for towing artillery and heavy equipment)
					\item 5x M35 Deuce and a Half Trucks (for specialized transport roles)
				\end{itemize}
				\item Engineer Company:
				\begin{itemize}
					\item 10x M4 Sherman Dozer Tanks (for earthmoving and obstacle clearing)
					\item 5x M2 Combat Engineering Vehicles (for construction and demolition tasks)
					\item 15x M3 Half-Tracks (for transport of engineering personnel and equipment)
					\item 5x Bailey Bridges (portable, pre-fabricated bridges for river crossings)
				\end{itemize}
				\item Medical Company:
				\begin{itemize}
					\item 20x WC54 Ambulances (for medical evacuation)
					\item 10x M37 Dodge Trucks (for medical equipment and personnel transport)
					\item 5x M43 Ambulance Vans (for field hospital support)
					\item 5x Portable Surgical Units (for on-site surgeries and trauma care)
				\end{itemize}
				\item Ordnance Company:
				\begin{itemize}
					\item 10x M26 Tank Recovery Vehicles (for recovery and repair of armored vehicles)
					\item 5x M32B1 Armored Recovery Vehicles (for battlefield recovery operations)
					\item 10x M3 Half-Tracks with maintenance kits (for on-field repair of vehicles and weapons)
				\end{itemize}
				\item Supply Company:
				\begin{itemize}
					\item 20x M35 Deuce and a Half Trucks (for transport of food, water, and ammunition)
					\item 10x Fuel Tankers (for refueling vehicles and equipment)
					\item 5x Water Purification Units (for providing clean water to troops)
					\item 10x Portable Refrigeration Units (for preserving perishable supplies)
				\end{itemize}
			\end{itemize}
	\end{itemize}

	\textbf{Command Element (CE)}: The CE is responsible for overall command and control of the QRF, including planning, coordination, intelligence, communications, and liaison with other units and commands:
	\begin{itemize}
		\item 1x Command Group (approx. 250 - 300 personnel)
		\begin{itemize}
			\item 10x M3 Command Half-Tracks (for mobile command and control)
			\item 5x M20 Utility Cars (for rapid command mobility)
			\item 5x M38 Jeeps (for liaison and quick communication)
		\end{itemize}
		\item 1x Headquarters Company
		\begin{itemize}
			\item 5x Mobile Command Posts (for establishing field headquarters)
			\item 10x M3 Half-Tracks (for HQ personnel transport)
			\item 5x M35 Deuce and a Half Trucks (for transport of HQ equipment)
		\end{itemize}
		\item 1x Communications Platoon
		\begin{itemize}
			\item 10x SCR-299 Radio Trucks (for long-range communications)
			\item 5x AN/TRC-1 Radio Sets (for command network communication)
			\item 10x Signal Corps M38 Jeeps (equipped with short-range radios for battlefield communication)
		\end{itemize}
		\item 1x Intelligence Platoon
		\begin{itemize}
			\item 5x M3 Half-Tracks (for intelligence personnel transport)
			\item 5x SCR-694 Radio Sets (for intelligence communication)
			\item 5x M38 Jeeps (for reconnaissance missions)
		\end{itemize}
		\item 1x Liaison Section
		\begin{itemize}
			\item 5x M38 Jeeps (for liaison officers' rapid deployment)
			\item 5x M20 Utility Cars (for secure and fast liaison)
		\end{itemize}
		\item 1x Military Police Platoon
		\begin{itemize}
			\item 5x M8 Armored Cars (for security and convoy escort duties)
			\item 10x M38 Jeeps (for patrol and security operations)
			\item 5x M3 Half-Tracks (for transport of MP personnel)
		\end{itemize}
	\end{itemize}	
}

\clearpage

\printbibliography

\end{document}
